\documentclass[11pt]{article}
% \usepackage{times}
\usepackage{palatino}
\usepackage{colortbl}
\usepackage{enumitem}

\renewcommand{\baselinestretch}{1.2} 
\setlength{\topmargin}{-1.0in}
\setlength{\textwidth}{6.5in}
\setlength{\oddsidemargin}{0.0in}
\setlength{\textheight}{10.1in}

\newlength{\pagewidth}
\setlength{\pagewidth}{6.5in}
\pagestyle{empty}

\def\pp{\par\noindent}

\special{papersize=8.5in,11in}


\begin{document}
\noindent 
Tony Li \\
CSE150 \\
HW01b \\\\
%=========================================Problem 1=====================================================
	\textbf{\underline{Problem 4:}}


	Boolean algebra operations can be expressed as arithmetic operations mod 2. Let 1 represent 

	true, and 0 false.

	( a ) Show that A $\wedge$ B = ( A $\bullet$ B mod 2 ).\\

	\begin{minipage}{0.5\textwidth}
		\begin{tabular}{||c c c||} 
 		\hline
 		A & B & A $\wedge$ B\\ [0.5ex] 
 		\hline\hline
 		T & T & T\\ 
 		\hline
 		T & F & F\\
 		\hline
 		F & T & F\\
 		\hline
 		F & F & F\\
 		\hline
		\end{tabular}
	\end{minipage}
	\begin{minipage}{0.5\textwidth}
		\begin{tabular}{||c c c||} 
 		\hline
 		A & B & A $\bullet$ B mod 2\\ [0.5ex] 
 		\hline\hline
 		1 & 1 & 1\\ 
 		\hline
 		1 & 0 & 0\\
 		\hline
 		0 & 1 & 0\\
 		\hline
 		0 & 0 & 0\\
 		\hline
		\end{tabular}
	\end{minipage}\\\\

	( b ) What is $\sim$A?\\

	\begin{tabular}{||c c c c||} 
 		\hline
 		A & $\sim$A & A+1 & A+1 mod 2\\ [0.5ex] 
 		\hline\hline
 		1 & 0 & 2 & 0\\ 
 		\hline
 		1 & 0 & 2 & 0\\
 		\hline
 		0 & 1 & 1 & 1\\
 		\hline
 		0 & 1 & 1 & 1\\
 		\hline
		\end{tabular}\\\\

	( c ) What is A $\vee$ B? (Use De Morgan's laws.)\\

	\begin{enumerate}
		\item A $\vee$ B
		\item $\sim$ ( $\sim$A $\wedge$ $\sim$B) \textit{[by De Morgan's Laws]}
		\item $\sim$( $\sim$A $\bullet\sim$B )$\bmod$2  \textit{[by Problem 4a]}
		\item ( ( A + 1 ) $\bmod$2 $\bullet$ ( B + 1 )$\bmod$2 + 1 )$\bmod$2 \textit{[by Problem 4b]}
	\end{enumerate}





\end{document}