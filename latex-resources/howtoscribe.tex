% Begin your document as any latex file
\documentclass{article}
% Use package cse150
\usepackage{cse150}

% The following are just for the sake of this document.  Ignore them
% for the purposes of generating scribe notes.
% --------------------------------------------------------------------
\newcommand{\pc}[1]{{\tt\begin{description} \item [\TT] #1
\end{description}}}

\newcommand{\com}[1]{$\backslash$#1}
\newcommand{\comone}[2]{$\backslash$#1\{#2\}}
\newcommand{\comtwo}[3]{$\backslash$#1\{#2\}\{#3\}}
\newcommand{\comthree}[4]{$\backslash$#1\{#2\}\{#3\}\{#4\}}
\newcommand{\comfour}[5]{$\backslash$#1\{#2\}\{#3\}\{#4\}\{#5\}}
\newcommand{\comfive}[6]{$\backslash$#1\{#2\}\{#3\}\{#4\}\{#5\}\{#6\}}

\newcommand{\sh}[2]{\com{#1} & $#2$ \\}
\newcommand{\shn}[2]{\com{#1} & #2 \\}

\newenvironment{todolist}{\begin{description}}{\end{description}}
\newcommand{\todoitem}[2]{\item {\em #1} \hfill \\ #2}
% ---------------------------------------------------------------------

% Begin the lecture with a line like the following:
% This replaces the usual \begin{document} that a latex
% file begins with.
\begin{lecture}{0}{How to Do Scribe Notes}{Seth Gilbert}

% Now summarize the lecture.  This should be like an abstract
% for the lecture, broken down into a few bullet points.
\begin{summary}
  % The first item of each \sumitem is the title of the part of the
  % lecture, and the second part is a description (i.e. abstract) of
  % that part.
  \sumitem{Using cse150.sty}{In this section, we
  describe the basics of using the cse150.sty latex style file, focuing
  on how to setup your documents.}
  \sumitem{Macros in cse150.sty}{In this section, we describe how
  to use the various macros in cse150.sty to write good scribe notes.}
  \sumitem{Quality scribe notes}{We list the most common mistakes in
  scribe notes, and provide tips on how to avoid them.}
\end{summary}

% Now organize your lecture into sections.
\section{Using cse150.sty}
\seclabel{general-instructions}

The {\bf cse150.sty} file contains a set of macros to help you typeset
your problem sets and scribe notes.  This file contains an example of
how to use these macros to produce scribe notes.  Use the following
steps to set up your document:
\begin{enumerate}
\item Begin your document as you would any latex document:
  \pc{\comone{documentstyle}{article}}
\item Use the cse150 package:
  \pc{\comone{usepackage}{cse150}}
\item Add any other preamble information.
\item Begin the lecture specify the lecture you are scribing:
  \pc{\comfour{begin}{lecture}{{\it Lecture Number}}{{\it Lecture
  Name}}{{\it Scribe Name}}}
  For example:
  \pc{\comfour{begin}{lecture}{5}{Advanced Algorithms}{Joe
  Programmer}}
  If someone other than Michael Bender gave the lecture, use the
  following:
  \pc{\comfive{begin}{guestlecture}{5}{Advanced Algorithms}{Joe
  Programmer}{Sam Professor}}
\item Create a summary of the lecture:
  \pc{\comone{begin}{summary}}
  \pc{\TT \comtwo{sumitem}{Basic Terms}{We define some basic terms}}
  \pc{\TT \comtwo{sumitem}{Lemmas}{We state a few lemmas}}
  \pc{\TT \comtwo{sumitem}{Proofs}{We prove some lemmas}}
  \pc{\comone{end}{summary}}
\item Describe the lecture.
\item End the document:
  \pc{\comone{end}{lecture}}
  \pc{\com{theend}}
  Don't forget the last line!
\end{enumerate}

\section{Macros in cse150.sty}

In this section we discuss some of the common macros available in
cse150.sty.  First, in \secref{thm-proof}, we discuss various theorem
and proof environments.  Then, in \secref{label-ref}, we describe how
to use the label and reference commands.  Finally, in
\secref{commands}, we present a table of useful macros for common
symbols.  Please see the cse150.sty file itself for a number of macros
not included here.  There are many useful macros that are no
documented here, but will make writing good latex documents much
easier.

\subsection{Theorem/Proof Environments}
\seclabel{thm-proof}

The cse150.sty file defies many of the usual theorem environments:
theorem, corollary, lemma, observation, proposition, definition,
claim, fact, assumption.  So, for example, one present a theorem as
follows: 
\pc{\comone{begin}{theorem}} 
\pc{\TT Statement of the theorem goes here.}  
\pc{\comone{end}{theorem}} 
Environments for proofs are similarly provided:
\pc{\comone{begin}{proof}}
\pc{\TT This is why my theorem is true.}
\pc{\comone{end}{proof}}
Together, these two will lead to something that looks like:
\begin{theorem}
  Statement of the theorem goes here.
\end{theorem}
\begin{proof}
  This is why my theorem is true.
\end{proof}

Similarly, environments for the following are provided:
proof-sketch, proof-of-lemma, proof-idea, proof-attempt, proofof,
remark.

\subsection{Labels and References}
\seclabel{label-ref}

In this section, we go over how to use various label and reference
commands.  We want to be produce a document containing all the scribe
notes for the semester at the end of the term.  To this end, we want
all references to be relative (and not absolute), to be correct (and
refer to a label of the correct type), and to be unique (so as not to
create difficulties in combining all the documents together.

To this end, the cse150.sty file contains a set of commands for labels
and references.  In order to label a section, using the following
command:
\pc{\comone{seclabel}{mysection}}
This produces a unique label that contains the lecture number, and
also indicates that it is a label of a section.  In order to reference
this label, use the following command:
\pc{\comone{secref}{mysection}}
This will produce the following text:
\pc{\secref{label-ref}}
(if \emph{mysection} is actually that section).  This is slightly
different than the usual \com{label} and \com{ref} commands, but
essentially the same.  Label and reference commands exist for the
following types: \\
\hfill \\
\begin{tabular}{lll}
Label Command & Reference Command & Example \\
\hline
\com{seclabel} & \com{secref} & Section 1\\
& \com{secreftwo} & Sections 1 and 2\\
& \com{secrefthree} & Sections 1, 2, and 3\\
& \com{secreffour} & Sections 1, 2, 3, and 4\\
\com{applabel} & \com{appref} & Appendix 1\\
\com{figlabel} & \com{figref} & Figure 1 \\
& \com{figreftwo} & Figures 1 and 2 \\
& \com{figrefthree} & Figures 1, 2, and 3 \\
\com{tablabel} & \com{tabref} & Table 1 \\
\com{stlabel} & \com{stref} & Step 1 \\
\com{thmlabel} & \com{thmref} & Theorem 1 \\
\com{lemlabel} & \com{lemref} & Lemma 1 \\
& \com{lemreftwo} & Lemmas 1 and 2 \\
& \com{lemfrefthree} & Lemmas 1, 2, and 3 \\
\com{corlabel} & \com{corref} & Corollary 1 \\
\com{equlabel} & \com{equref} & Equation 1 \\
& \com{equreftwo} & Equations 1 and 2 \\
\com{inequlabel} & \com{inequref} & Inequality 1 \\
& \com{inequreftwo} & Inequalities 1 and 2 \\
\com{invlabel} & \com{invref} & Invariant 1 \\
\com{deflabel} & \com{defref} & Definition 1 \\
\com{proplabel} & \com{propref} & Proposition 1 \\
\com{caselabel} & \com{caseref} & Case 1 \\
& \com{casereftwo} & Cases 1 and 2 \\
\com{lilabel} & \com{liref} & Line 1 \\
\end{tabular} \\
\hfill \\
In special cases where no appropriate label and reference command
exist, you can use:
\pc{\comone{genlabel}{mylabel}}
\pc{\comone{genref}{mylabel}}
\noindent to get a generic label and reference.  Please avoid using these,
however, when appropriate label and reference commands exist.

\subsection{Math and Other Symbols}
\seclabel{commands}

A number of useful commands are defined.  See the following table: \\
\hfill \\
\begin{tabular}{ll}
Command & Result \\
\hline
\sh{ihat}{\ihat}
\sh{jhat}{\jhat}
\sh{OP}{\OP}
\sh{OPprime}{\OPprime}
\sh{abs\{var\}}{\abs{var}}
\sh{bigO}{\bigO}
\sh{set\{a,b,c\}}{\set{a,b,c}}
\sh{half}{\half}
\sh{Pr\{x\}}{\Pr{x}}
\sh{Exp\{x\}}{\Exp{x}}
\sh{implies}{\implies}
\sh{sizeof\{thing\}}{\sizeof{thing}}
\sh{setof\{stuff\}}{\setof{stuff}}
\sh{reals}{\reals}
\sh{integers}{\integers}
\sh{naturals}{\naturals}
\sh{rationals}{\rationals}
\sh{complex}{\complex}
\sh{norm\{x\}}{\norm{x}}
\sh{card\{x\}}{\card{x}}
\sh{floor\{x\}}{\floor{x}}
\sh{ceil\{x\}}{\ceil{x}}
\sh{ang\{x\}}{\ang{x}}
\sh{sbrace\{x\}}{\sbrace{x}}
\sh{cbrace\{x\}}{\cbrace{x}}
\sh{paren\{x\}}{\paren{x}}
\sh{Var}{\Var}
\sh{expect\{x\}}{\expect{x}}
\sh{expectsq\{x\}}{\expectsq{x}}
\sh{variance\{x\}}{\variance{x}}
\sh{choose\{x\}\{y\}}{\choose{x}{y}}
\sh{percent\{95\}}{\percent{95}}
\sh{twodots}{\twodots}
\sh{transpose}{\transpose}
\sh{amortized\{x\}}{\amortized{x}}
\sh{cases\{one case\}}{\cases{one case}}
\sh{cif\{x\}}{\cif{x}}
\sh{cwhen\{x\}}{\cwhen{x}}
\sh{cotherw}{\cotherw}
\end{tabular}

\section{Writing Guidelines for Scribe Notes \\ and Other Technical
  Writing}

\subsection{Extra Writing Help}

Writing Guides -- Read the latex manual and \emph{The Elements of
Style} by Strunk and White. For more help with technical writing see
``Writing Resources on the World Wide Web'' at \\
\pc{http://web.mit.edu/uaa/www/writing/links/}

\subsection{Organization of Scribe Notes}

\begin{todolist}
\sumitem{Boldface for Organization}{Carefully choose the titles of sections,
subsections, subsubsections, and paragraphs.  The reader should learn
all proof ideas and lecture topics from reading the boldface only and
ignoring the main text.}

\todoitem{Location of Topic Sentences}{In technical writing the topic sentence
of each paragraph should be located at the beginning of the paragraph.
For scribe notes there are no exceptions.  Many writers want to put
the topic sentence in the middle or the end of the paragraph, and the
writing quality suffers.}

\todoitem{Topic Paragraphs}{A numbered section (e.g., section, subsection)
should begin with a topic paragraph, which explains the content of the
section. A topic paragraph is not necessary in an unnumbered section.}

\todoitem{Figures and Long Figure Captions}{Please use as many figures as you
can. Do not be afraid to write long figure captions.  The reader
should understand much of the lecture just from viewing the figures
and reading the captions.  A figure caption begins with a sentence
fragment followed by a period.  Only full sentences follow.}

\todoitem{Write Clear Latex}{Latex is source code and consequently should be
clear and readable. Avoid personalized environments.}

\todoitem{Text in Figure Captions Versus Body of Notes}{In the figure
caption explain what the figure represents. In the body of this scribe
notes say what the figure means.  For example, in a lecture on task
scheduling, we may have a figure showing a directed a cyclic graph.
The figure caption reads as follows: ``Example of a precedence graph.
Circles represent tasks and edges represent dependencies.''}
% I didn't finish writing this yet...

\end{todolist}

\subsection{Sentence Structure}

\begin{todolist}
\todoitem{Omission of Needless Words}{Write clearly with short
sentences and simple constructions.  If the word is not absolutely
necessary, then remove it.}

\todoitem{Use of Present Tense}{Simplify verb tenses and use the
present tense as much as possible.  Avoid future and conditional
tenses and the subjunctive mood when possible. For example, do not
write ``we will show'' when ``we show'' suffices; do not write ``we
can establish'' when ``we establish'' suffices.}

\todoitem{Active Verbs}{English is a language in which verbs contain
most of the force of the sentence. Therefore use the active voice.  Do
not say: ``It is established in this lecture...''; say: ``This lecture
establishes....''}

\todoitem{Strong Subjects of Sentences}{In English the subject also contains
force.  Therefore as above, avoid saying ``It is....''}

\todoitem{Precision}{The words ``this'' and ``that'' need a noun after
them.  For example, do not say: ``This is why we use... period'' Say:
``This impossibility result is why we use....''}

\end{todolist}

\subsection{Use of Mathematics}

\begin{todolist}

\todoitem{Punctuation in Centered Equations}{Centered equations are
grammatically part of the surrounding text.  Consequently, they still
need punctuation.  A period or comma at the end of a centered equation
is preceded by a small amount of space (`\,' in latex) before it.}

\todoitem{Use of Eqnarray}{Use eqnarray frequently.  In centered equations you
should never have more than one `$=$' (or `$<$' `$>$') per line.}

\todoitem{Spelling out Numbers}{An English rule is that numbers less
then or equal to twelve are spelled out, whereas larger numbers are
written with Arabic numerals.  This guideline is modified for
technical writing: If you do arithmetic on a number, represent it with
numerals.  Otherwise follow the English rule.}

\todoitem{Representation of Fractions}{In running text you should
generally use $1/2$ and avoid $\frac{1}{2}$ (and similarly for other
fractions).  In centered mathematics you should generally use
$\frac{1}{2}$.}

\end{todolist}

\subsection{When to Ignore the Rules}

Ignore the above rules until your ideas are written. Then as you edit
your text, apply all rules.  Do not try to write perfect text from the
outset.

\end{lecture}
\theend
