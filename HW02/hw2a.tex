\documentclass[11pt]{article}
% \usepackage{times}
\usepackage{palatino}
\usepackage{colortbl}

\usepackage{amssymb}

\usepackage{tikz}
\usetikzlibrary{shapes.geometric, arrows}
\tikzstyle{startstop} = [rectangle, rounded corners, minimum width=3cm, minimum height=1cm,text centered, draw=black, fill=red!30]
\tikzstyle{io} = [trapezium, trapezium left angle=70, trapezium right angle=110, minimum width=3cm, minimum height=1cm, text centered, draw=black, fill=blue!30]
\tikzstyle{process} = [rectangle, minimum width=3cm, minimum height=1cm, text centered, draw=black, fill=orange!30]
\tikzstyle{decision} = [diamond, minimum width=3cm, minimum height=1cm, text centered, draw=black, fill=green!30]
\tikzstyle{arrow} = [thick,->,>=stealth]

\renewcommand{\baselinestretch}{1.2} 
\setlength{\topmargin}{-1.0in}
\setlength{\textwidth}{6.5in}
\setlength{\oddsidemargin}{0.0in}
\setlength{\textheight}{10.1in}

\newlength{\pagewidth}
\setlength{\pagewidth}{6.5in}
\pagestyle{empty}

\def\pp{\par\noindent}

\special{papersize=8.5in,11in}


\begin{document}
\noindent 
Tony Li \\
CSE150 \\
HW02a \\\\

%============================================Problem 1======================================================
	\noindent\textbf{\underline{Problem 1}}\\
	For each of the following statements about sets determine whether it is always true ( also provide 
	an example), or only sometimes true ( also provide an example and counterexample). Please 
	provide an explanation.

	\begin{enumerate}
		\item $A \in P(A)$ \\\\
			Always true. \\\\
				\underline{True Example:} $\emptyset \in P(\emptyset)  \Longleftrightarrow \emptyset \in \{\emptyset\} $ \\
				The power set is the set of all subsets, so A will always be an element in P(A).
				
		\item $A \subseteq P(A)$\\\\
			 Sometimes True. \\\\
				\underline{False Example:} $\{1,2\}$ is not a subset of $\{\emptyset , \{1\} , \{2\} , \{1, 2\}\}$       

		\item $(|A| \leq |B| \Rightarrow (A \subseteq B))$\\\\
			Sometimes True. \\\\
				\underline{True Example:} $|\{4, 5, 6\}| \leq |\{4, 5, 6, 7\}|$ and $\{4, 5, 6\} \subseteq \{4, 5, 6, 7\}$ is true.\\
				\underline{False Example:} $|\{1, 2, 3\}| \leq |\{4, 5, 6, 7\}|$ and $\{1, 2, 3\} \subseteq \{4, 5, 6, 7\}$ is not true.\\
		\item $( A \subseteq B ) \Rightarrow ( |A| \leq |B| )$	
			Always true. \\\\
				\underline{True Example:} $\{1, 2\} \subseteq \{1, 2, 3\}$ and $|\{1, 2\}| \leq |\{1, 2, 3\}|$ \\
				If A is a subset of B, then the number of elements in A is less than or equal to B. 
	\end{enumerate}
\ \\\\\\\\\\\\\\\\\\\\\\
%============================================Problem 2======================================================
	\noindent\textbf{\underline{Problem 2}}\\
	Find the smallest two finite sets A and B for each of the four conditions.\\
\textit{Note:} The smallest sets may not be unique.

	\begin{enumerate}
		\item $A \in B, A \subseteq B$, and $P(A) \subseteq B$

			$A = \emptyset , B = \{ \emptyset \}$\\
		\item $( \mathbb{N} \cap A ) \in A, B \subset A,$ and $P(B) \subseteq A.$

			$A = \{ \emptyset, \{ \emptyset \} \} , B = \emptyset$\\
		\item $A \subseteq (P (P(B)) - P(A)$.

			$A = \emptyset, B = \emptyset$\\
		\item $A \supseteq P (P(B)) - P(A)$.

			$A = \{ \emptyset \} , B = \emptyset$
	
	\end{enumerate}

%============================================Problem 3======================================================
	\noindent\textbf{\underline{Problem 3}}\\
	Prove or disprove (by providing a counterexample) each of the following properties of binary relations:\\
	Let S(A) be the symmetric closure of set A. Let T(A) be the transitive closure of set A. For every binary relation R, 

	\begin{enumerate}
		\item $T(S(R)) \subseteq S(T(R))$ \\
			\textbf{False.}

			$R = \{(1 , 3)\}$ 

			$S(R) = \{(3, 1)\}$ 

			$T(R) = \{(1,3)\}$

			$T(S(R)) = \{(1, 3) , (3, 1) , (1, 1) , (3, 3)\}$

			$S(T(R)) = \{(1, 3) , (3, 1)\}$

		\item $S(T(R)) \subseteq T(S(R))$ \\
			\textbf{True}

			$T(R) = \{ (x, y) , (y, z) \in R$ $|$ $(x, z) \in R \}$ \\
			$S(T(R)) = \{ (x, y) , (y, z) (x, z) \in R$ $|$ $(y, x), (z, y), (z, x) \in R\}$ \\
			\\
			$S(R) = \{ (x, y) , (y, z) \in R$ $|$ $(y, x), (z, y) \in R \}$ \\
			$T(S(R)) = \{ (x, y) , (y, z) (y, x), (z, y) \in R$ $|$ $(x, z), (z, x) \in R\}$ \\\\

			These cases will end up being equivalent to each other, except when T(R) fails to add an edge to R. That's when S(T(R)) becomes a subset of T(S(R)).
	\end{enumerate}

%============================================Problem 4======================================================
	\noindent\textbf{\underline{Problem 4}}\\
	How many reflexive binary relations are there on $S$ x $S$? How many symmetric relations? Explain. \\
	$Bonus:$ How many equivalence relations are there on $S$ x $S$? Explain. \\

	\underline{Reflexive Relations} \\

	$S$ x $S$ = $\{ x \in S$ and $y \in S$ $|$ $(x, y) \in R\}$\\

	The reflexive relations in $S$ x $S$ can be shown with a table: \\\\

	\begin{tabular}{|c |c c c c c c c|} 
 		\hline
 		X & 0 & 1 & 2 & 3 &... & n - 1 & n\\ [0.5ex] 
 		\hline
 		0 & (0, 0) &  \cellcolor{red!25}(0, 1) &  \cellcolor{red!25}(0, 2) &  \cellcolor{red!25}(0, 3) &  \cellcolor{red!25}(0, ...) &  \cellcolor{red!25}(0, n - 1) &  \cellcolor{red!25}(0, n)\\ 
 		\hline
		1 & \cellcolor{red!25}(1, 0) & (1, 1) &  \cellcolor{red!25}(1, 2) &  \cellcolor{red!25}(1, 3) &  \cellcolor{red!25}(1, ...) &  \cellcolor{red!25}(1, n - 1) &  \cellcolor{red!25}(1, n)\\ 
 		\hline
		2 & \cellcolor{red!25}(2, 0) & \cellcolor{red!25}(2, 1) & (2, 2) &  \cellcolor{red!25}(2, 3) &  \cellcolor{red!25}(2, ...) &  \cellcolor{red!25}(2, n - 1) &  \cellcolor{red!25}(2, n)\\ 
 		\hline
		3 & \cellcolor{red!25}(3, 0) & \cellcolor{red!25}(3, 1) & \cellcolor{red!25}(3, 2) & (3, 3) &  \cellcolor{red!25}(3, ...) &  \cellcolor{red!25}(3, n - 1) &  \cellcolor{red!25}(3, n)\\ 
 		\hline
		... & \cellcolor{red!25}(..., 0) & \cellcolor{red!25}(..., 1) & \cellcolor{red!25}(..., 2) & \cellcolor{red!25}(..., 3) & (..., ...) &  \cellcolor{red!25}(..., n - 1) &  \cellcolor{red!25}(..., n)\\ 
 		\hline
		n - 1 & \cellcolor{red!25}(n - 1, 0) & \cellcolor{red!25}(n - 1, 1) & \cellcolor{red!25}(n - 1, 2) & \cellcolor{red!25}(n - 1, 3) & \cellcolor{red!25}(n - 1, ...) & (n - 1, n - 1) &  \cellcolor{red!25}(n - 1, n)\\ 
 		\hline
		n & \cellcolor{red!25}(n, 0) & \cellcolor{red!25}(n, 1) & \cellcolor{red!25}(n, 2) & \cellcolor{red!25}(n, 3) & \cellcolor{red!25}(n, ...) & \cellcolor{red!25}(n, n - 1) & (n, n)\\ 
 		\hline
	\end{tabular}\\

	The number of elements in all possible ordered pairs that can be reflexive is the length of $S$ x $S$

	minus the length of the set containing the ordered pairs, (x, y), such that x = y because any set 

	can be reflexive so long as the set with x = y is a subset. So the power set of the length of the 

	set would be $2^{n^2 - n}$
\newpage
	\underline{Symmetric Relations} \\
	
	$S$ x $S$ = $\{ x \in S$ and $y \in S$ $|$ $(x, y) \in R\}$\\

	The symmetric relations in $S$ x $S$ can be shown with a table: \\\\

	\begin{tabular}{|c |c c c c c c c|} 
 		\hline
 		X & 0 & 1 & 2 & 3 &... & n - 1 & n\\ [0.5ex] 
 		\hline
 		0 & \cellcolor{blue!25}(0, 0) & (0, 1) & (0, 2) & (0, 3) & (0, ...) & (0, n - 1) & (0, n)\\ 
 		\hline
		1 & \cellcolor{red!25}(1, 0) & \cellcolor{blue!25}(1, 1) & (1, 2) & (1, 3) & (1, ...) & (1, n - 1) & (1, n)\\ 
 		\hline
		2 & \cellcolor{red!25}(2, 0) & \cellcolor{red!25}(2, 1) & \cellcolor{blue!25}(2, 2) & (2, 3) & (2, ...) & (2, n - 1) & (2, n)\\ 
 		\hline
		3 & \cellcolor{red!25}(3, 0) & \cellcolor{red!25}(3, 1) & \cellcolor{red!25}(3, 2) & \cellcolor{blue!25}(3, 3) & (3, ...) & (3, n - 1) & (3, n)\\ 
 		\hline
		... & \cellcolor{red!25}(..., 0) & \cellcolor{red!25}(..., 1) & \cellcolor{red!25}(..., 2) & \cellcolor{red!25}(..., 3) & \cellcolor{blue!25}(..., ...) & (..., n - 1) & (..., n)\\ 
 		\hline
		n - 1 & \cellcolor{red!25}(n - 1, 0) & \cellcolor{red!25}(n - 1, 1) & \cellcolor{red!25}(n - 1, 2) & \cellcolor{red!25}(n - 1, 3) & \cellcolor{red!25}(n - 1, ...) & \cellcolor{blue!25}(n - 1, n - 1) & (n - 1, n)\\ 
 		\hline
		n & \cellcolor{red!25}(n, 0) & \cellcolor{red!25}(n, 1) & \cellcolor{red!25}(n, 2) & \cellcolor{red!25}(n, 3) & \cellcolor{red!25}(n, ...) & \cellcolor{red!25}(n, n - 1) & \cellcolor{blue!25}(n, n)\\ 
 		\hline
	\end{tabular}\\\\
		
	The number of elements in all possible ordered pairs that can be in a symmetric relation is the

	 sum of the length of the set containing (x,y) where x = y, and the length of the set containing
	
	 an (y, x) for every (x, y) where x != y divided by 2. For every (x, y), there must be a (y, x), so

	dividing by 2 would get rid of redundancies. \\

	To calculate how many cells are in the highlighted area given a set $S$, you would use $\Sigma n$ 

	which is the same as $\frac{n(n+1)}{2}$. The total number of symmetric relations would then be

	the length of the power set of the set, which would be $2^{\frac{n(n+1)}{2}}$. \\

***Collaborated with Sean Chu, Raymond Wu, and David Song

	

	 
	
\end{document}






