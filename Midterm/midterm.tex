\documentclass[12pt]{exam}
\usepackage{lastpage}
\setlength\parindent{0pt}
\renewcommand{\questionlabel}{(\thequestion)}

\usepackage[letterpaper,margin=1in]{geometry}
\def\zz{{\mathbb Z}}
\def\rr{{\mathbb R}}
\def\qq{{\mathbb Q}}
\def\nn{{\mathbb N}}
\def\cc{{\mathbb C}}

\usepackage{colortbl}

\newcommand{\Z}{{\mathbb Z}{}}
\newcommand{\N}{{\mathbb N}{}}
\newcommand{\Q}{{\mathbb Q}{}}
\newcommand{\R}{{\mathbb R}{}}
\newcommand{\C}{{\mathbb C}{}}

\def\myline{\rule{12cm}{.01in}}
\def\seq{{\subseteq}}

\usepackage{tabularx}
\usepackage{enumitem}
\usepackage{latexsym}
\usepackage{epsfig}
\usepackage{xspace}
\usepackage{color}
\usepackage{dsfont}
\usepackage{amsmath}
\usepackage{amssymb}
\usepackage{wasysym}
\usepackage{wrapfig}
\usepackage{tikz}
\usepackage{verbatim}
\usepackage{graphicx}
\usepackage{etoolbox}
\AtBeginEnvironment{choices}{%
   \par\medskip\begin{minipage}{\linewidth}}
\makeatletter
\AtEndEnvironment{choices}{%
   \if@correctchoice \endgroup \fi%
   \end{minipage}}
\makeatother

\usetikzlibrary{arrows}

\newcommand*\circled[1]{\tikz[baseline=(char.base)]{
            \node[shape=circle,draw,inner sep=2pt] (char) {#1};}}

\newcommand{\mab}[1]{{\small \sf \color{blue}{MAB: #1}}}

\newcommand{\Mod}[1]{\ (\text{mod}\ #1)}
\newcommand{\twodots}{\mathinner{\ldotp\ldotp}}
\renewcommand{\epsilon}{\varepsilon}
\newcommand{\litlow}[1]{\mathord{\mathcode`\-="702D\sf #1\mathcode`\-="2200}}
\newcommand{\lit}[1]{\ensuremath{\litlow{#1}}}

\newcommand{\defn}[1] {{\textit{\textbf{\myboldmath #1\/}}}}
\newcommand{\myboldmath}{\boldmath}

\newcommand{\junk}[1]{}

\newtheorem{theorem}            {Theorem}
\newtheorem{claim}[theorem]     {Claim}

\newenvironment{mcquestion}[3]
{\begin{minipage}{\linewidth}
\question[#1][#2] #3
\begin{choices}}
{\end{choices}
\end{minipage}}



% Compact itemize and enumerate.  Note that they use the same counters and
% symbols as the usual itemize and enumerate environments.
\def\compactify{\itemsep=0pt \topsep=0pt \partopsep=0pt \parsep=0pt}
\let\latexusecounter=\usecounter
\newenvironment{CompactItemize}
  {\def\usecounter{\compactify\latexusecounter}
   \begin{itemize}}
  {\end{itemize}\let\usecounter=\latexusecounter}
\newenvironment{CompactEnumerate}
  {\def\usecounter{\compactify\latexusecounter}
   \begin{enumerate}}
  {\end{enumerate}\let\usecounter=\latexusecounter}

% Avoid line breaks before citations (\cite) and references (\ref)
\let\latexcite=\cite
\def\cite{\nolinebreak\latexcite}
\let\latexref=\ref
\def\ref{\nolinebreak\latexref}


%%%%%%%%%%


\newcommand{\algorithm}[1]{
{\rm
\begin{center}
\begin{tabbing}
while \= while \= while \= while \= while \= while \= while \= while \= \kill
#1
\end{tabbing}
\end{center}
}
}

\newcounter{problemctr}
\addtocounter{problemctr}{0}

% total of counters
\newcounter{totalpoints}

\newcommand{\docounters}[2]{
    \newcounter{#1}
    \addtocounter{#1}{#2}

    % summing counters:
    \addtocounter{totalpoints}{#2}
}

%%%%%%%%%%%%%%%%%%%%%%%%%%
%%  problem numbers and point values
%%%%%%%%%%%%%%%%%%%%%%%%%%









%%%%%%%%%%%%%%%%%%%%%%%%%%%%%%%%%%%%%%%%%%%%%%%%% BEGIN ATTEMPT TO MAKE SIMPLER
%%%%%%%%%%%%%%%%%%%%%%%%%%%%%%%%%%%%%%%%%%%%%%%%% BEGIN ATTEMPT TO MAKE SIMPLER
%%%%%%%%%%%%%%%%%%%%%%%%%%%%%%%%%%%%%%%%%%%%%%%%% BEGIN ATTEMPT TO MAKE SIMPLER
%%%%%%%%%%%%%%%%%%%%%%%%%%%%%%%%%%%%%%%%%%%%%%%%% BEGIN ATTEMPT TO MAKE SIMPLER
%%%%%%%%%%%%%%%%%%%%%%%%%%%%%%%%%%%%%%%%%%%%%%%%% BEGIN ATTEMPT TO MAKE SIMPLER
%%%%%%%%%%%%%%%%%%%%%%%%%%%%%%%%%%%%%%%%%%%%%%%%% BEGIN ATTEMPT TO MAKE SIMPLER
%%%%%%%%%%%%%%%%%%%%%%%%%%%%%%%%%%%%%%%%%%%%%%%%% BEGIN ATTEMPT TO MAKE SIMPLER
%%%%%%%%%%%%%%%%%%%%%%%%%%%%%%%%%%%%%%%%%%%%%%%%% BEGIN ATTEMPT TO MAKE SIMPLER
% How to define a new problem %NOT YET WORKING
\newcommand{\newproblem}[5]{
    \newcommand{#1} {
        \newpage
        \addtocounter{problemctr}{1}
        \noindent
        {\bf Problem \theproblemctr.  (#3\xspace points)}

        \smallskip
        #5
    }
    \addtocounter{totalpoints}{#3}
    \newcommand{#2}{#3}
}

%% to add text later, write \latertextnow

%%%%%%%%%%%%%%%%%%%%%%%%%%%%%%%%%%%%%%%%%%%%%%%%% END ATTEMPT TO MAKE SIMPLER
%%%%%%%%%%%%%%%%%%%%%%%%%%%%%%%%%%%%%%%%%%%%%%%%% END ATTEMPT TO MAKE SIMPLER
%%%%%%%%%%%%%%%%%%%%%%%%%%%%%%%%%%%%%%%%%%%%%%%%% END ATTEMPT TO MAKE SIMPLER
%%%%%%%%%%%%%%%%%%%%%%%%%%%%%%%%%%%%%%%%%%%%%%%%% END ATTEMPT TO MAKE SIMPLER
%%%%%%%%%%%%%%%%%%%%%%%%%%%%%%%%%%%%%%%%%%%%%%%%% END ATTEMPT TO MAKE SIMPLER
%%%%%%%%%%%%%%%%%%%%%%%%%%%%%%%%%%%%%%%%%%%%%%%%% END ATTEMPT TO MAKE SIMPLER
%%%%%%%%%%%%%%%%%%%%%%%%%%%%%%%%%%%%%%%%%%%%%%%%% END ATTEMPT TO MAKE SIMPLER



 % points for signing
\newcounter{namesign}
\addtocounter{namesign}{2}
\addtocounter{totalpoints}{\thenamesign}

% Name, # points, time

\docounters{choices}{48}
\docounters{powersets}{18}
\docounters{relation}{12}


\docounters{unused}{0}


\begin{document}

{\bf
\noindent
CSE 150 --  Foundations of Computer Science: Honors\hfill             Midterm Exam\\
Michael Bender\hfill      Tuesday, November 1, 2016\\
}
\rule{\linewidth}{.01in}
\begin{center}
{\bf ~~~}
\end{center}
\fbox{{\Huge{\bf TEST VERSION: 1}}}\\
\medskip
\bigskip

Name: Tony Li ID \#: \rule{2.5in}{.01in}\\[\bigskipamount]

\noindent {\bf INSTRUCTIONS:}
\begin{itemize}
\item Unless otherwise stated, your answers should be at most $1$ or $2$ sentences (excluding work.)
\item This is a closed book, closed notes exam.
\item Check to see that you have \pageref{LastPage} pages including this cover and scratch pages. %!!! check num of pages and change it at the end
\item Read all the problems before starting work.
\item Think before you write.
\item If you leave a question blank or write just ``I do not know,'' you get 25\% automatically.
\item Good luck!!
\end{itemize}

Academic integrity is expected of all students at all times, whether
in the presence or absence of members of the faculty.

Understanding this, I declare that I shall not give, use, or receive
unauthorized aid in this examination.  I have been warned that if I
am caught cheating (either receiving or giving unauthorized aid) I
will get a ``Q'' grade for this course, and a letter will be sent to
the Committee on Academic Standing and Appeals (CASA) requesting
that an academic dishonesty notation be placed on my transcript.
Further action against me may also be taken.
\bigskip

\noindent
Signature \footnote{No ``I dunno'' points for leaving this blank. \smiley} Tony Li



\nopagebreak


\begin{center}
\begin{tabular}{||c|c|c||} \hline
Problem&Score&Maximum\\ \hline
signature && \thenamesign \\ \hline
1&& \thechoices\\ \hline
2&& \thepowersets\\ \hline
3&& \therelation\\ \hline
Total& &\thetotalpoints\\ \hline
\end{tabular}
\end{center}

\bigskip

% !TEX root =  midterm.tex

\newpage

\addtocounter{problemctr}{1}

{\bf
Problem \theproblemctr.  (\thechoices\xspace points)}
\\

{\bf Write your choice on the left of the question number.}


\begin{center}
{\bf Part A}\\
\medskip
(In this part each question has a {\bf single} correct answer. 4 points each.)
\end{center}
\bigskip

\begin{questions}

\question Which of the following is a proposition:	
\begin{choices}
\choice A proposition.
\choice Choose me.
\choice Don't you want to choose the last option?
\choice Surely you're joking.
\choice Trust none of the above answers.
\end{choices}

\circled{D}

\vspace{1.8in}

\question Let $S$ be an infinite set and suppose we have $S=S_1\times S_2 \times S_3\times \dots \times S_n$, where $n\in \N$. Which of the following is correct?
\begin{choices}
\choice At least one of the sets $S_i$ is an infinite set.
\choice At least one of the sets $S_i$ is an finite set.
\choice At most one of the sets $S_i$ can be an infinite set.
\choice At most one of the sets $S_i$ is a finite set.
\choice None of the above.
\end{choices}

\circled{A}
\newpage

\question If $f(x) = \lfloor x \rfloor$, over which of the following
domains and co-domains is it bijective? (Answers in the form (Domain,
Co-domain).)\footnote{The notation $\lfloor x\rfloor$ (the ``floor''
  of $x$) means the largest integer less than or equal to $x$.  Thus,
  $\lfloor 2\rfloor=2$ and $\lfloor 1.9\rfloor=1$.}
\begin{choices}
\choice ($\mathbb{R}, \mathbb{R}$) 
\choice ($\mathbb{Z}, \mathbb{N}$) 
\choice ($\mathbb{Z}, \mathbb{Z}$) 
\choice ($\mathbb{Z}, \mathbb{Q}$) 
\choice ($\mathbb{Q}, \mathbb{R}$) 
\end{choices}

\circled{C}

\vspace{1.8in}

\question Recall the block-unstacking problem from class. 
We start with a stack of $n$ blocks, and the objective is to unstack them fully. 
The \emph{unstacking operation} takes a stack of size $s$, and splits it into two stacks, of size $x$ and $s-x$. 
The unstacking cost is $x(s-x) + 1$.  What is the minimum cost to unstack all of the blocks? 
\begin{choices}
\choice $n^2-1$
\choice $n(n-2) + 2$
\choice $n(n-1)/2$ 
\choice $n(n+1)/2$
\choice $(n+2)(n-1)/2$
\end{choices}

\circled{E}
\newpage

\question If $a \equiv 0 \Mod 3,\ a \equiv 1 \Mod 4,\ a \equiv 2 \Mod 5$, which of the following could $a$ equal?
\begin{choices}
\choice 27
\choice 256
\choice 257
\choice 357
\choice 2357
\end{choices}

\circled{D}

\vspace{1.8in}

\question We say that a relation $R$ is \defn{coreflexive} if for any $x$ and $y$: 
$$ %\forall x,y\,\,\, 
(x,y)\in R \, \Rightarrow\,  x=y.$$  
Let $S = \{a, b, c\}$.  Which of the following relations on $S$ is
{\bf both} reflexive and coreflexive?
\begin{choices}
\choice $\emptyset$
\choice $\{(a,c),(b,c),(c,a),(c,b)\}$
\choice $\{(a,a),(b,b),(c,c)\}$
\choice $\{(a,a),(b,b)\}$
\choice $\{(a,a),(b,b),(c,c),(a,c),(a,b)\}$
\end{choices}

\circled{C}
\newpage

\question Which of the following formulae correctly represents ``The set $A$ has exactly two elements''?
\begin{choices}
\choice $\exists x,y\in A,\,\, x\neq y$.
\choice $\forall x,y\in A,\,\, x\neq y$.
\choice $\forall z\in A,\,\, \exists x,y\in A,\ (x\neq y)\land \big( (z=x) \lor (z=y) \big)$.
\choice $\exists x,y\in A,\,\, (x\neq y)\land \big(\forall z\in A,\ (z\neq x) \Rightarrow (z=y)\big)$.
\choice $\exists x,y\in A,\,\, \forall z\in A,\,\, (z=x) \lor (z=y)$.
\end{choices}

\circled{D}

\vspace{1.8in}


\question Let $T$ be the set of all reflexive binary relations on $\N$, the set of natural numbers. Then $T$ is uncountable. We prove it by contradiction.\\

Assume we have a enumeration $T = \{R_1,\ R_2,\ \dots\}$. Then we create a relation $R$, such that $(i,i)\in R,\ \forall i\in \N$. And for all $i$ and $j$ where $i\neq j$, $(i,j)\in R \iff (i,j)\not\in R_{\lvert i-j \rvert}$. Then $R\not\in T$ but $R$ is reflexive. So $T$ is uncountable.\\

Which of the following judgments is correct?
\begin{choices}
\choice The conclusion is wrong.
\choice The proof is wrong because the assumed enumeration is not clear.
\choice The proof is wrong because $R$ is not even reflexive.
\choice The proof is wrong because $R$ could still equal some $R_k\in T$.
\choice The proof is right.
\end{choices}

\end{questions}

\circled{E}
\newpage

\begin{center}
{\bf Part B}\\
\medskip
(In this part each question may have {\bf one or more} correct answers. 4 points each.\\ If your answer is a {\bf nonempty strict subset} of the correct answer, you get 2.5 points.)
\end{center}
\bigskip

\begin{questions}

\question Let $R_1$ and $R_2$ be two binary relations. 

$R_1$ is said to be \defn{contained} in $R_2$ if for any $x$ and $y$, $\, (x,y)\in R_1 \Rightarrow (x,y)\in R_2$. 

The relation ``is contained in'' (which is a relation on relations)
is:
\begin{choices}
\choice Reflexive
\choice Symmetric
\choice Transitive
\end{choices}
\circled{A},\circled{C}
\vspace{1.8in}

\question Let $A = \Big\{\emptyset,\{\emptyset\},\big\{\emptyset,\{\emptyset\} \big\} \Big\}$. Which of the following are (is) true?
\begin{choices}
\choice $\Big\{\big\{\{\emptyset\}\big\}\Big\} \in P(A)$.
\choice $A \times \{\emptyset\} = \{{\emptyset}\}$.
\choice $A \times \emptyset = \emptyset$.
\end{choices}

\circled{C}
\vspace{0.2in}


\newpage

\question Let $A$ be an infinite set and $B$ be a countable set. Which pair(s) of sets of the following must have a bijection?
\begin{choices}
\choice $A$ and $B$
\choice $A$ and $P(B)$
\choice $A$ and $A \cap B$
\choice $A$ and $A \cup B$
\choice $A$ and $A \times B$
\end{choices}

\circled{D}
\vspace{0.2in}

\vspace{1.8in}


\question Which of the following statements are (is) true?
\begin{choices}
\choice $\N\times\Q$ is countable.
\choice The set of all binary relations on a countably infinite set is countable.
\choice $P(\Q)$ is countable.
\choice The union of countably many countable sets can be uncountable.
\choice The intersection of finitely many uncountable sets can be countably infinite.
\end{choices}
\circled{A}
\end{questions}

% !TEX root =  midterm.tex

\newpage
\addtocounter{problemctr}{1}

{\bf
Problem \theproblemctr.  (\thepowersets\xspace points)}
\\

For each of the following statements about sets, 
state whether it is {\bf always} true (provide an {\bf example}),
{\bf sometimes} true (provide an {\bf example} and {\bf counterexample}), or {\bf never} true (provide a {\bf counterexample}).

\bigskip

\begin{enumerate}[label=(\arabic*),itemsep=.2in]

\item $S\in P(P(S))$
\textbf{\hfill always \quad\quad\quad \circled{sometimes} \quad\quad\quad never}
\renewcommand{\arraystretch}{3}
\begin{tabular}{ll}
\textbf{Example}: & $S= \emptyset$\\
\textbf{Counterexample}: & $S= \{1,2\}$\\
\end{tabular}


\item $P(S\cap T) = P(S)\cap P(T)$
\textbf{\hfill \circled{always} \quad\quad\quad sometimes \quad\quad\quad never}
\renewcommand{\arraystretch}{3}
\begin{tabular}{ll}
\textbf{Example}: & $S= \emptyset$\\
& $T= \emptyset$\\
\textbf{Counterexample}: & $S=$ \underline{\hbox to 60mm{}}\\
& $T=$ \underline{\hbox to 60mm{}}\\
\end{tabular}
\newpage

\item $P(S - T) = P(S) - P(T)$
\textbf{\hfill always \quad\quad\quad sometimes \quad\quad\quad \circled{never}}
\renewcommand{\arraystretch}{3}
\begin{tabular}{ll}
\textbf{Example}: & $S=$ \underline{\hbox to 60mm{}}\\
& $T=$ \underline{\hbox to 60mm{}}\\
\textbf{Counterexample}: & $S= \emptyset$\\
& $T= \emptyset$\\
\end{tabular}

\end{enumerate}


% !TEX root =  midterm.tex

\newpage
\addtocounter{problemctr}{1}

{\bf
Problem \theproblemctr.  (\therelation\xspace points)}
\\

Let's do some counting. Given a finite set $S$, where $\lvert S \rvert= n$, fill in the blanks below and give a {\bf one-sentence} justification for each.

\vspace{.2in}

\begin{enumerate}[label=(\arabic*),itemsep=.3in]
\item
Total number of binary relations on $S$: $2^{n^2}$ \\
Number of binary relations = $|P(S x S)| = 2^{|SxS|} = 2^{|S|^2} = 2^{n^2}$
\item
Number of reflexive binary relations on $S$: $2^{n^2-n}$ \\\\
Number of reflexive binary relations $P (SxS$ - set containing binary relations (a,b) where (a,b) $\in$ $SxS$ and $a=b)|$  \\
\begin{tabular}{|c |c c c c c c c|} 
 		\hline
 		X & 0 & 1 & 2 & 3 &... & n - 1 & n\\ [0.5ex] 
 		\hline
 		0 & (0, 0) &  \cellcolor{red!25}(0, 1) &  \cellcolor{red!25}(0, 2) &  \cellcolor{red!25}(0, 3) &  \cellcolor{red!25}(0, ...) &  \cellcolor{red!25}(0, n - 1) &  \cellcolor{red!25}(0, n)\\ 
 		\hline
		1 & \cellcolor{red!25}(1, 0) & (1, 1) &  \cellcolor{red!25}(1, 2) &  \cellcolor{red!25}(1, 3) &  \cellcolor{red!25}(1, ...) &  \cellcolor{red!25}(1, n - 1) &  \cellcolor{red!25}(1, n)\\ 
 		\hline
		2 & \cellcolor{red!25}(2, 0) & \cellcolor{red!25}(2, 1) & (2, 2) &  \cellcolor{red!25}(2, 3) &  \cellcolor{red!25}(2, ...) &  \cellcolor{red!25}(2, n - 1) &  \cellcolor{red!25}(2, n)\\ 
 		\hline
		3 & \cellcolor{red!25}(3, 0) & \cellcolor{red!25}(3, 1) & \cellcolor{red!25}(3, 2) & (3, 3) &  \cellcolor{red!25}(3, ...) &  \cellcolor{red!25}(3, n - 1) &  \cellcolor{red!25}(3, n)\\ 
 		\hline
		... & \cellcolor{red!25}(..., 0) & \cellcolor{red!25}(..., 1) & \cellcolor{red!25}(..., 2) & \cellcolor{red!25}(..., 3) & (..., ...) &  \cellcolor{red!25}(..., n - 1) &  \cellcolor{red!25}(..., n)\\ 
 		\hline
		n - 1 & \cellcolor{red!25}(n - 1, 0) & \cellcolor{red!25}(n - 1, 1) & \cellcolor{red!25}(n - 1, 2) & \cellcolor{red!25}(n - 1, 3) & \cellcolor{red!25}(n - 1, ...) & (n - 1, n - 1) &  \cellcolor{red!25}(n - 1, n)\\ 
 		\hline
		n & \cellcolor{red!25}(n, 0) & \cellcolor{red!25}(n, 1) & \cellcolor{red!25}(n, 2) & \cellcolor{red!25}(n, 3) & \cellcolor{red!25}(n, ...) & \cellcolor{red!25}(n, n - 1) & (n, n)\\ 
 		\hline
	\end{tabular}\\
\item
Number of symmetric binary relations on $S$: $2^{\frac{n(n+1)}{2}}$ \\\\
Number of symmetric relations = the size of the power set of all binary relations $(a,b)$ in $SxS$ that have a $(b,a)$ in $SxS$

\begin{tabular}{|c |c c c c c c c|} 
 		\hline
 		X & 0 & 1 & 2 & 3 &... & n - 1 & n\\ [0.5ex] 
 		\hline
 		0 & \cellcolor{blue!25}(0, 0) & (0, 1) & (0, 2) & (0, 3) & (0, ...) & (0, n - 1) & (0, n)\\ 
 		\hline
		1 & \cellcolor{red!25}(1, 0) & \cellcolor{blue!25}(1, 1) & (1, 2) & (1, 3) & (1, ...) & (1, n - 1) & (1, n)\\ 
 		\hline
		2 & \cellcolor{red!25}(2, 0) & \cellcolor{red!25}(2, 1) & \cellcolor{blue!25}(2, 2) & (2, 3) & (2, ...) & (2, n - 1) & (2, n)\\ 
 		\hline
		3 & \cellcolor{red!25}(3, 0) & \cellcolor{red!25}(3, 1) & \cellcolor{red!25}(3, 2) & \cellcolor{blue!25}(3, 3) & (3, ...) & (3, n - 1) & (3, n)\\ 
 		\hline
		... & \cellcolor{red!25}(..., 0) & \cellcolor{red!25}(..., 1) & \cellcolor{red!25}(..., 2) & \cellcolor{red!25}(..., 3) & \cellcolor{blue!25}(..., ...) & (..., n - 1) & (..., n)\\ 
 		\hline
		n - 1 & \cellcolor{red!25}(n - 1, 0) & \cellcolor{red!25}(n - 1, 1) & \cellcolor{red!25}(n - 1, 2) & \cellcolor{red!25}(n - 1, 3) & \cellcolor{red!25}(n - 1, ...) & \cellcolor{blue!25}(n - 1, n - 1) & (n - 1, n)\\ 
 		\hline
		n & \cellcolor{red!25}(n, 0) & \cellcolor{red!25}(n, 1) & \cellcolor{red!25}(n, 2) & \cellcolor{red!25}(n, 3) & \cellcolor{red!25}(n, ...) & \cellcolor{red!25}(n, n - 1) & \cellcolor{blue!25}(n, n)\\ 
 		\hline
	\end{tabular}\\\\

\end{enumerate}

\newpage




{\bf \begin{center}
    \newpage 
    Scratch Paper
    \newpage 
    Scratch Paper
    \newpage 
    Scratch Paper
    \newpage 
    Scratch Paper
\end{center}}

\begin{comment}



\end{comment}



\end{document}
