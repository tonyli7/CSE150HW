% !TEX root =  midterm.tex

\newpage

\addtocounter{problemctr}{1}

{\bf
Problem \theproblemctr.  (\thechoices\xspace points)}
\\

{\bf Write your choice on the left of the question number.}


\begin{center}
{\bf Part A}\\
\medskip
(In this part each question has a {\bf single} correct answer. 4 points each.)
\end{center}
\bigskip

\begin{questions}

\question Which of the following is a proposition:	
\begin{choices}
\choice A proposition.
\choice Choose me.
\choice Don't you want to choose the last option?
\choice Surely you're joking.
\choice Trust none of the above answers.
\end{choices}

\circled{D}

\vspace{1.8in}

\question Let $S$ be an infinite set and suppose we have $S=S_1\times S_2 \times S_3\times \dots \times S_n$, where $n\in \N$. Which of the following is correct?
\begin{choices}
\choice At least one of the sets $S_i$ is an infinite set.
\choice At least one of the sets $S_i$ is an finite set.
\choice At most one of the sets $S_i$ can be an infinite set.
\choice At most one of the sets $S_i$ is a finite set.
\choice None of the above.
\end{choices}

\circled{A}
\newpage

\question If $f(x) = \lfloor x \rfloor$, over which of the following
domains and co-domains is it bijective? (Answers in the form (Domain,
Co-domain).)\footnote{The notation $\lfloor x\rfloor$ (the ``floor''
  of $x$) means the largest integer less than or equal to $x$.  Thus,
  $\lfloor 2\rfloor=2$ and $\lfloor 1.9\rfloor=1$.}
\begin{choices}
\choice ($\mathbb{R}, \mathbb{R}$) 
\choice ($\mathbb{Z}, \mathbb{N}$) 
\choice ($\mathbb{Z}, \mathbb{Z}$) 
\choice ($\mathbb{Z}, \mathbb{Q}$) 
\choice ($\mathbb{Q}, \mathbb{R}$) 
\end{choices}

\circled{C}

\vspace{1.8in}

\question Recall the block-unstacking problem from class. 
We start with a stack of $n$ blocks, and the objective is to unstack them fully. 
The \emph{unstacking operation} takes a stack of size $s$, and splits it into two stacks, of size $x$ and $s-x$. 
The unstacking cost is $x(s-x) + 1$.  What is the minimum cost to unstack all of the blocks? 
\begin{choices}
\choice $n^2-1$
\choice $n(n-2) + 2$
\choice $n(n-1)/2$ 
\choice $n(n+1)/2$
\choice $(n+2)(n-1)/2$
\end{choices}

\circled{E}
\newpage

\question If $a \equiv 0 \Mod 3,\ a \equiv 1 \Mod 4,\ a \equiv 2 \Mod 5$, which of the following could $a$ equal?
\begin{choices}
\choice 27
\choice 256
\choice 257
\choice 357
\choice 2357
\end{choices}

\circled{D}

\vspace{1.8in}

\question We say that a relation $R$ is \defn{coreflexive} if for any $x$ and $y$: 
$$ %\forall x,y\,\,\, 
(x,y)\in R \, \Rightarrow\,  x=y.$$  
Let $S = \{a, b, c\}$.  Which of the following relations on $S$ is
{\bf both} reflexive and coreflexive?
\begin{choices}
\choice $\emptyset$
\choice $\{(a,c),(b,c),(c,a),(c,b)\}$
\choice $\{(a,a),(b,b),(c,c)\}$
\choice $\{(a,a),(b,b)\}$
\choice $\{(a,a),(b,b),(c,c),(a,c),(a,b)\}$
\end{choices}

\circled{C}
\newpage

\question Which of the following formulae correctly represents ``The set $A$ has exactly two elements''?
\begin{choices}
\choice $\exists x,y\in A,\,\, x\neq y$.
\choice $\forall x,y\in A,\,\, x\neq y$.
\choice $\forall z\in A,\,\, \exists x,y\in A,\ (x\neq y)\land \big( (z=x) \lor (z=y) \big)$.
\choice $\exists x,y\in A,\,\, (x\neq y)\land \big(\forall z\in A,\ (z\neq x) \Rightarrow (z=y)\big)$.
\choice $\exists x,y\in A,\,\, \forall z\in A,\,\, (z=x) \lor (z=y)$.
\end{choices}

\circled{D}

\vspace{1.8in}


\question Let $T$ be the set of all reflexive binary relations on $\N$, the set of natural numbers. Then $T$ is uncountable. We prove it by contradiction.\\

Assume we have a enumeration $T = \{R_1,\ R_2,\ \dots\}$. Then we create a relation $R$, such that $(i,i)\in R,\ \forall i\in \N$. And for all $i$ and $j$ where $i\neq j$, $(i,j)\in R \iff (i,j)\not\in R_{\lvert i-j \rvert}$. Then $R\not\in T$ but $R$ is reflexive. So $T$ is uncountable.\\

Which of the following judgments is correct?
\begin{choices}
\choice The conclusion is wrong.
\choice The proof is wrong because the assumed enumeration is not clear.
\choice The proof is wrong because $R$ is not even reflexive.
\choice The proof is wrong because $R$ could still equal some $R_k\in T$.
\choice The proof is right.
\end{choices}

\end{questions}

\circled{E}
\newpage

\begin{center}
{\bf Part B}\\
\medskip
(In this part each question may have {\bf one or more} correct answers. 4 points each.\\ If your answer is a {\bf nonempty strict subset} of the correct answer, you get 2.5 points.)
\end{center}
\bigskip

\begin{questions}

\question Let $R_1$ and $R_2$ be two binary relations. 

$R_1$ is said to be \defn{contained} in $R_2$ if for any $x$ and $y$, $\, (x,y)\in R_1 \Rightarrow (x,y)\in R_2$. 

The relation ``is contained in'' (which is a relation on relations)
is:
\begin{choices}
\choice Reflexive
\choice Symmetric
\choice Transitive
\end{choices}
\circled{A},\circled{C}
\vspace{1.8in}

\question Let $A = \Big\{\emptyset,\{\emptyset\},\big\{\emptyset,\{\emptyset\} \big\} \Big\}$. Which of the following are (is) true?
\begin{choices}
\choice $\Big\{\big\{\{\emptyset\}\big\}\Big\} \in P(A)$.
\choice $A \times \{\emptyset\} = \{{\emptyset}\}$.
\choice $A \times \emptyset = \emptyset$.
\end{choices}

\circled{C}
\vspace{0.2in}


\newpage

\question Let $A$ be an infinite set and $B$ be a countable set. Which pair(s) of sets of the following must have a bijection?
\begin{choices}
\choice $A$ and $B$
\choice $A$ and $P(B)$
\choice $A$ and $A \cap B$
\choice $A$ and $A \cup B$
\choice $A$ and $A \times B$
\end{choices}

\circled{D}
\vspace{0.2in}

\vspace{1.8in}


\question Which of the following statements are (is) true?
\begin{choices}
\choice $\N\times\Q$ is countable.
\choice The set of all binary relations on a countably infinite set is countable.
\choice $P(\Q)$ is countable.
\choice The union of countably many countable sets can be uncountable.
\choice The intersection of finitely many uncountable sets can be countably infinite.
\end{choices}
\circled{A}, \circled{E}
\end{questions}
