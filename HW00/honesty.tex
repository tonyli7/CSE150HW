%% File history
%% 7/16/13 Michael Bender created document
%% 7/17/13 MB incorporated suggestions for improvement from Joe Mitchell
%% 2/2/14 MB updates for CSE 350 
%% 8/25/14 MB updates for CSE 150
%% 1/26/15 MB edits for CSE 215 and incorporates suggestions from Steve Skiena
%% 1/23/16 MB edits for CSE 350 and takes some of Don Porter's edits.
%% 8/28/16 MB edits for CSE 150 
%% hopefully lots of people will update and use this assignment :-)


\documentclass[11pt]{article}
% \usepackage{times}
\usepackage{palatino}

\usepackage{enumitem}

\renewcommand{\baselinestretch}{1.2} 
\setlength{\topmargin}{-0.5in}
\setlength{\textwidth}{5.5in}
\setlength{\oddsidemargin}{0.5in}
\setlength{\textheight}{9.1in}

\newlength{\pagewidth}
\setlength{\pagewidth}{6.5in}
\pagestyle{empty}

\def\pp{\par\noindent}

\special{papersize=8.5in,11in}


\begin{document}

\begin{flushleft}
   $\copyright$  2016~~Michael A. Bender
\end{flushleft}
\centerline{\bf CSE 150 -- 
Honors Foundations of Computer Science  Fall 2016}
\medskip
\centerline{Academic Honesty Review}
\bigskip
\bigskip

The point of this assignment is to examine issues of academic
integrity and academic dishonesty.

You should hand in your problem set \emph{online} using
blackboard. \emph{No late problem sets accepted.}

Latex is a wonderful and powerful scientific word processor. The first
homework must be done in Latex. \newline For the Macintosh, install
Mactex and use Texshop.  For linux, install texlive-full. Then you can
use linux programs such as texworks, texmaker, or gummi.  For Windows,
install Miktex and use WinEdt, texmaker, or some other editor. You can
also use online latex programs such as Overleaf or Sharelatex.

For drawing figures, I recommend using xfig, Ipe, inkscape, the Mac
tool Omnigraffle. You can also use more general presentation tools
such as Powerpoint, Keynote, or Libreoffice.

This writeup includes input from Joe Mitchell, Don Porter, and Steve
Skiena.  I welcome further suggestions from other students and
professors.


\section*{Academic Honesty}

These are bullet points from the course procedures document:

\begin{itemize}[noitemsep]

\item I take academic honesty \emph{very} seriously.  Infractions have
  serious consequences---generally an F in the course (or worse).

\item It is \emph{your} responsibility to ensure that you understand
  what constitutes academic dishonesty.

\item Representing another person's work as your own is always wrong.
  It is wrong in this course. It is wrong in your
  profession.\footnote{Plagiarizing can end your career, regardless of
    your skill level, your educational credentials, or how hard you
    have worked up until this point. } It is wrong in life. It is
  wrong. Period.


\item Always cite! If you work with multiple people, cite with whom you worked.

\item Copying (or approximately copying) a solution from the web or
  someone else's solution and putting in your problem set is
  plagiarism \emph{even if you cite your source}.

\item Sharing any part of your write-up (latex, PDF, postscript,
  figures, or hard copy) is academic dishonesty and invites
  plagiarism.  Your own write-up is private information and should not
  be shared.

\item You may be able to find solutions to some of the homework
  problems on the web or from more senior students. It is academic
  dishonesty to search for and use such solutions in preparing your
  own write-up for the assignment, and it is plagiarism to copy such
  solutions and to submit as your own (even if you cite). 

\item You can work together to solve problems, but you must write up
  your own solutions, writing only those ideas and answers that you
  personally fully understand, and stating in your write-up with whom
  you worked to obtain the solution.

\item If you are in doubt about whether or not you are permitted to
  use particular source materials, you should obtain written
  permission from the professor, in advance of your submission. Such
  permission is best requested and obtained by email.

\item {\em It is academically dishonest to hand in a solution that you
    don't understand.}


\end{itemize}
 
\bigskip\bigskip

\newcounter{problemctr}

\addtocounter{problemctr}{1}
\bigskip
\noindent
$\underline{\rm Problem\ \theproblemctr}$\pp 
%
Explain why we let students work together to solve problems, as long
as the students cite their collaborators. Explain why working together
is not academically dishonest in this course.

\addtocounter{problemctr}{1}
\bigskip
\noindent
$\underline{\rm Problem\ \theproblemctr}$\pp 
%
Explain why it is important to your professional development to
struggle with a problem that you cannot solve quickly. In other words,
the instructor deliberately assigns homework he knows you will likely
have to think about for days or weeks to solve. What do you expect to
learn from this experience?

\addtocounter{problemctr}{1}
\bigskip
\noindent
$\underline{\rm Problem\ \theproblemctr}$\pp 
%
Explain why, although it is ok to work with other students, it is
plagiarism to share and/or copy other write-ups.  Give an example of
collaboration that is academically honest. Give another example of
collaboration that is academically dishonest.

\addtocounter{problemctr}{1}
\bigskip
\noindent
$\underline{\rm Problem\ \theproblemctr}$\pp 
%
Explain why it is academically dishonest to share your solution set
with another student.  Explain how you could get burned from just
sharing your writeup even if you do not copy yourself.

\addtocounter{problemctr}{1}
\bigskip
\noindent
$\underline{\rm Problem\ \theproblemctr}$\pp Explain why copying (or
approximately copying) solutions from the web (or another source) is
plagiarism, even if you cite your source.\footnote{In this course I
  ask that you \emph{not} to scour the web for solutions to the
  homework problems. If there are class topics that you don't
  understand from reading the course materials come talk to me. But
  even I did let you search the web to find solutions, this would
  \emph{still} be plagiarism. Why? }
 
\addtocounter{problemctr}{1} \bigskip
\noindent
$\underline{\rm Problem\ \theproblemctr}$\pp 
%
Explain why it is better for your grade to leave a question blank,
rather than search for answers on the web.  (Hint: calculate
approximately how much a homework problem is worth to your raw score
versus an exam question. Feel free to include the risk-benefit
analysis of getting caught.)


% MAB: I'm commenting out this problem, because I'd rather students
% not use the internet to help them solve problems at all.
%
%\addtocounter{problemctr}{1}
%\bigskip
%\noindent
%$\underline{\rm Problem\ \theproblemctr}$\pp
%Suppose that you are completely stuck about how to approach a problem. 
%Give an example of using the internet to help that is academically honest. 
%Give an example of using the internet that is academically dishonest. 
%

\addtocounter{problemctr}{1}
\bigskip
\noindent
$\underline{\rm Problem\ \theproblemctr}$\pp 
%
Imagine that you are employed at a major software company, say Google,
Facebook, or Microsoft, and commit code into a product that you copied
from a website. Explain the potential risks to both you and the
company if this action is discovered by the owners of the code.

\addtocounter{problemctr}{1}
\bigskip
\noindent
$\underline{\rm Problem\ \theproblemctr}$\pp
%
Please speculate on why we decided to make a problem set on academic
honesty.

\addtocounter{problemctr}{1}
\bigskip
\noindent
$\underline{\rm Problem\ \theproblemctr}$\pp 
%
How much time did this writeup take you, including the time it took to
learn latex.



\end{document}

