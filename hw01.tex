\documentclass[12pt]{article} % sets size of whole document (still not sure how it scales with specified font-size)

\renewcommand{\baselinestretch}{1.2} % sets line spacing to 1.4
\setlength{\topmargin}{-0.5in} % changes top margin by -.05 inches
\setlength{\textwidth}{6.5in} % self-explanatory
\setlength{\oddsidemargin}{0.0in} % pushes text to the right by param in inches
\setlength{\textheight}{7.7in} % self-explanatory
\begin{document}
\begin{flushleft} % move text to left-most of page
Tony Li  \\
CSE150 \\
HW01
\end{flushleft}

\noindent
\textbf{Problem 1:}
\\
\noindent
Students are allowed to work together because it allows students to gain different approaches
to difficult problems. Working together is not academically dishonest because sometimes solutions
can only be obtained through the meshing of different ideas. 
\\
\\
\textbf{Problem 2:}
\\
\noindent
It's important to struggle with a problem I can't solve quickly because it happens all the time.
Being able to be patient and really use all the resources available to me is a necessary skill
that will be useful, not just in CS, but in other classes as well. 
\\
\\
\textbf{Problem 3:}
\\
\noindent
It's plagiarism to copy write-ups because it's not your own work. You're stealing someone else's work 
and claiming as though you understood the material. It's also wrong to share your work because of the 
risk of the receiver to steal your work. A collaboration that is academically honest would be discussing
the different ways of implementing an algorithm using Java with other students and then using the 
extra insight you gained from the discussion to code the algorithm. A collaboration that is dishonest 
woud be one person writing all the code and then everyone else copies it and submits it as their own. 
\\
\\
\textbf{Problem 4:}
\\
\noindent
It's academically dishonest to share your solution with another student because you're giving 
another student the opportnity to cheat and you can even get backstabbed by the receiving
student, since he/she can claim that you were the one who plagiarized.
\\
\\
\textbf{Problem 5:}
\\
\noindent
Copying solutions from the web or another source is plagiarism even if you cite the source because
it's no different from copying the write-up of another student and giving credit to his/her name.
\\
\\
\textbf{Problem 6:}
\\
\noindent
It is better to leave a question blank, rather than search for answers on the web because if that 
were the case, then you really didn't understand the question or the concepts required to do the
problem. Sure, you may get credit for the homework question, but it discourages you to actually
understand how to do the question, so when the final comes around, you're screwed since it's 
worth a large chunk of your grade.  
\\
\\
\textbf{Problem 7:}
\\
\noindent
It's an incredible risk because you will get fired, humiliate the company by branding them as plagiarizers,
and risk the company getting sued. You may also have a hard time finding a job, since the act of
plagiarism would go on your record.
\\
\\
\textbf{Problem 8:}
\\
\noindent
There is a problem set on academic honesty because there are many grey areas that may not be clear
to a student of what academic honesty is and you wouldn't want a student to plagiarize and then come
up with the lame excuse of "I didn't know it was plagiarism." Also, it would be a nice starting point to
get us familair with LaTeX.
\\
\\
\textbf{Problem 9:}
\\
\noindent
This took around two hours because I spent the first hour poking around your honesty.tex file to figure out
what certain things did and the next hour answering the questions. I was also distracted by the demon 
that told me to check my Facebook.

\end{document}