\documentclass[11pt]{article}
% \usepackage{times}
\usepackage{palatino}
\usepackage{colortbl}
\usepackage{amssymb}
\usepackage{tikz}
\usetikzlibrary{shapes.geometric, arrows}
\tikzstyle{startstop} = [rectangle, rounded corners, minimum width=3cm, minimum height=1cm,text centered, draw=black, fill=red!30]
\tikzstyle{io} = [trapezium, trapezium left angle=70, trapezium right angle=110, minimum width=3cm, minimum height=1cm, text centered, draw=black, fill=blue!30]
\tikzstyle{process} = [rectangle, minimum width=3cm, minimum height=1cm, text centered, draw=black, fill=orange!30]
\tikzstyle{decision} = [diamond, minimum width=3cm, minimum height=1cm, text centered, draw=black, fill=green!30]
\tikzstyle{arrow} = [thick,->,>=stealth]

\renewcommand{\baselinestretch}{1.2} 
\setlength{\topmargin}{-1.0in}
\setlength{\textwidth}{6.5in}
\setlength{\oddsidemargin}{0.0in}
\setlength{\textheight}{10.1in}

\newlength{\pagewidth}
\setlength{\pagewidth}{6.5in}
\pagestyle{empty}

\def\pp{\par\noindent}

\special{papersize=8.5in,11in}


\begin{document}
\noindent 
Tony Li \\
CSE150 \\
HW03b \\\\

%==============================================Problem 8====================================================
\noindent\textbf{\underline{Problem 8}}\\
	
	You have 10000 kilograms of pickles. Pickles are 99 percent water by volume. Water comprises 

	100 percent of the mass of the pickle. Time goes by, and you observe that some water has 

	evaporated. Now water comprises only 98 percent of the volume. What is the weight of the 

	pickles now?\\

	The weight of the pickles is $\frac{98}{99}$*10,000kg. If the volume of the pickles was V, then 

	.99V represents the volume of the pickles and also weighs 10,000 kg. If .1V is gone, then .98V

	is left, so $\frac{1}{99}$th of the volume of water disappeared, and thus, $\frac{1}{99}$th of 

	the mass is also gone. 

\	\\
%==============================================Problem 10===================================================
\noindent\textbf{\underline{Problem 10}}\\
	
	Prove the following using mathematical induction: 

	\begin{enumerate}
		\item $2n$ $\leq$ $2^n$

			Show that $2n$ $\leq$ $2^n$ is true.

			\underline{Proof:} Proof by induction on $n$\\

			Let $P(n)$ be the predicate that $2n$ $\leq$ $2^n$\\

			\textit{Base Case:} $n=1$, $2(1)$ $\leq$ $2^1$ is true, so P(1) holds. \\
			\textit{Inductive Assumption:} Assume $P(n)$ holds for $n = k$, that is, that $2k$ $\leq$ $2^k$ for some $k$.\\
			\textit{Inductive Step:} We want to establish P(k+1). If we substitute $k$ with $k+1$ we get: 

			\[2(k+1)  \leq 2^{k+1}\]

			This can be rewritten as:

			\[2k+2 \leq 2^k * 2\]
			\[2k+2 \leq 2^k + 2^k\]

			Since the inductive assumption is made and the base case was true, we show that $2k$, which is less 

			than or equal to $2^n$ added to $2(1)$ which is less than or equal to any $2^k$, we establish $P(k+1)$.\\

			Therefore, $P(n)$ holds for all $n$ by the principle of mathematical induction, thus proving the theorem. 
		\newpage
		\item $1+3+5+$ ... $+(2n-1)=n^2$

			Show that $1+3+5+$ ... $+(2n-1)=n^2$ is true.

			\underline{Proof:} Proof by induction on $n$\\

			Let $P(n)$ be the predicate that $1+3+5+$ ... $+(2n-1)=n^2$

			\textit{Base Case:} $n=1$, $1 = 1^2$ is true, so P(1) holds.
			\textit{Inductive Assumption:} Assume $P(n)$ holds for $n = k$, that is, that $1+3+5+$ ... $+(2k-1)=k^2$ for some $k$.\\
			\textit{Inductive Step:} We want to establish P(k+1). If we add $2k+1$ to both sides, we get:

			\[1+3+5+ ... +(2k-1)+(2k+1) = k^2+2k+1 = (k+1)^2\]
			
			Adding $2k+1$ comes from the inductive assumption and the rest follows by simplification.\\

			Therefore, $P(n)$ holds for all $n$ by the principle of mathematical induction, thus proving the theorem.
	

		\item $1^2+2^2+3^2+...+n^2 = \frac{(n)(n+1)(2n+1)}{6}$

			Show that $1^2+2^2+3^2+...+n^2 = \frac{(n)(n+1)(2n+1)}{6}$

			\underline{Proof:} Proof by induction on $n$\\

			Let $P(n)$ be the predicate that $1^2+2^2+3^2+...+n^2 = \frac{(n)(n+1)(2n+1)}{6}$

			\textit{Base Case:} $n=1$, $1^2 = \frac{(1)(1+1)(2(1)+1)}{6}$ is true, so P(1) holds.\\
			\textit{Inductive Assumption:} Assume $P(n)$ holds for $n = k$, that is, that 

			$1^2+2^2+3^2+...+k^2 = \frac{(k)(k+1)(2k+1)}{6}$\\

			\textit{Inductive Step:} We want to establish P(k+1). If we add $(k+1)^2$ to both sides, we get:

			\[1^2+2^2+3^2+...+k^2+(k+1)^2 = \frac{(k)(k+1)(2k+1)}{6} + (k+1)^2\]
			\[\frac{(k)(k+1)(2k+1)}{6} + k^2+2k+1\]
			\[\frac{(k+1)(k+2)(2k+3)}{6}\]

			Adding $(k+1)^2$ comes form the inductive assumption and the rest follows by simplification. \\
	
			Therefore, $P(n)$ holds for all $n$ by the principle of mathematical induction, thus proving the theorem.	
	\end{enumerate}
\newpage
%==============================================Problem 11===================================================
\noindent\textbf{\underline{Problem 11}}\\

	You have an $n$ x $m$ bar of chocolate. Your goal is to separate all of the squares of chocolate. The 

	way that you can break the chocolate is to take a single piece of chocolate (connected 

	component of squares) and break it along one horizontal or vertical line. What is the minimum 

	number of breaks necessary? Please prove your answer.

	Show that it takes $nm-1$ breaks to separate an $n$ x $m$ bar of chocolate.\\

	\underline{Proof:} Proof by strong induction on $N$\\

	Let "$P(N)$ $= N - 1$ breaks to separate the chocolate bar" be the predicate, where $N = nm$.

	\textit{Base Case:} $N=1$ P(1 x 1) = 0 is true. \\
	
	\textit{Inductive Assumption:} Assume that $\forall_k$ $\leq$ $N-1$, the number of breaks to

	separate the chocolate bar $= k - 1$\\

	\textit{Inductive Step:} We want to establish P(N). \\

	Given $N$ pieces of chocolate, it can be broken into two pieces, $N - k$ and $k$. So that means that 

	it takes $P(N-k) + P(k)+1$ breaks to completely separate the chocolate, or $N-k-1 + k-1 +1=N-1$ 

	by the inductive assumption.\\

	Therefore, $P(N)$ holds for all N by the principle of strong mathematical induction, thus proving the 

	theorem.\\


%==============================================Problem 12===================================================
\noindent\textbf{\underline{Problem 12}}\\

	You have an $n$ x $n$ checkerboard with an initial set of checkers placed on it. You are allowed to 

	add additional checkers under the following conditions: You can place a checker on a square 

	if two or more neighboring squares also have checkers on them. Neighboring cells are those

	above, below, to the left and to the right, as shown in Figure 2. As we showed in class, there

	are initial configurations of $n$ checkers that enable the entire board to be covered. Prove that 
	
	no configuration of $n-1$ checkers can let you cover the board.\\

	\underline{Proof:} Proof by induction on $t$\\

	Let $P(t)$ be the predicate, that is, that the perimeter formed by $n-1$ checkers will be less than 

	$4n$, or the perimeter of the checkerboard at $t$ timesteps.\\

	\textit{Base Case:} $t = 1$, 0 checkers cannot cover the 1 x 1 board. True. \\
	
	\textit{Inductive Assumption:} Assume $P(t)$ holds for t=k, that is, that the perimeter formed by $n-1$

	checkers $\leq$ $4n$.

	\newpage

	\textit{Inductive Step:} We want to establish $P(k+1)$. \\

	By placing $n-1$ checkers on the board, the perimeter formed will be less than or equal to 

	$4(n-1)$. With every timestep, the only way the perimeter will increase is if the checker is

	placed adjacent to only one checker or non-adjacent, which is rendered impossible by the

	rules. \\

	By the principle of mathematical induction, we prove that the checkerboard can't be covered

	with $n-1$ checkers. 

\end{document}